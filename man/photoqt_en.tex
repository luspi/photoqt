% Info: http://www.informatik-vollmer.de/software/latex2man-html.php
\documentclass[10pt]{article}
% Paquetes
\usepackage[latin1]{inputenc}
\usepackage{ae}
\usepackage{aecompl}
\usepackage[spanish]{babel}
\usepackage[pdftex,bookmarksopen,bookmarksnumbered]{hyperref}
\usepackage[nofancy]{latex2man}


\begin{document}

% Fecha y Version para LaTeX:
% $ date '+%Y-%m-%d'
\setDate{2018-06-20}
\setVersion{1.7.1}

\begin{Name}{1}{photoqt}{Lukas Spies}{\pkgname-\pkgversion}{photoqt -- A simple and attractive Image Viewer}
\Prog{PhotoQt} - A simple and attractive Image Viewer
\end{Name}

\section{Synopsis}
\noindent
\Prog{photoqt} \oArg{options} \oArg{filename}

\section{Description}
\noindent
\Prog{PhotoQt} is a fast and attractive image viewer developed for
different platforms. The application uses different engines display,
so it supports more than 80 types of image files. In addition, it
allows a fast and intuitive performance and management of images,
achieving to interact with the user through shortcuts totally
customizable.\MANbr
Thumbnails, slide shows, information on the Image (meta-data) and
other features make \Prog{PhotoQt} a powerful choice as the default
program for image viewing of your system.

\section{Options}
Generic options:
\begin{Description}
\item[\Opt{-h, --help}] Displays this help.
\item[\Opt{-v, --version}] Displays version information.
\end{Description}
Interaction with PhotoQt:
\begin{Description}
\item[\Opt{-o, --open}] Make PhotoQt ask for a new File.
\item[\Opt{-s, --show}] Shows PhotoQt (does nothing if already shown).
\item[\Opt{--hide}] Hides PhotoQt to system tray (does nothing if already
hidden).
\item[\Opt{-t, --toggle}] Toggle PhotoQt - hides PhotoQt if visible,
shows if hidden.
\item[\Opt{--thumbs, --no-thumbs}] Enable/Disable thumbnails.
\end{Description}
Start-up-only options:
\begin{Description}
\item[\Opt{--start-in-tray}] Start PhotoQt hidden to the system tray.
\item[\Opt{--standalone}] Create standalone PhotoQt, multiple instances
but no remote interaction possible.
\end{Description}
General Options:
\begin{Description}
\item[\Opt{--debug, --no-debug}] Switch on/off debug messages.
\item[\OptoArg{--export}{filename}] Export configuration to given filename.
\item[\OptoArg{--import}{filename}] Import configuration from given filename.
\end{Description}
Arguments:
\begin{Description}
\item[\oArg{filename}] File to open with PhotoQt.
\end{Description}

\section{Files}
\noindent
The application create and use the follow configuration directories:
\begin{description}
\item[\$HOME/.config/PhotoQt] Personal settings on Linux platforms.
Use the GUI interface in 'Settings' to modify them.
\item[\%USERPROFILE\%\Bs AppData\Bs Local\Bs PhotoQt] Personal settings
on Linux platforms. Use the GUI interface in 'Settings' to modify them.
\end{description}

\section{Author}
\noindent
Lukas Spies, developer, <\Email{Lukas@PhotoQt.org}>.\\
Web page project: <\URL{http://photoqt.org/}>.

\section{Bugs}
\noindent
Send a report of bugs in:
\begin{itemize}[*]
\item Email: <\Email{Lukas@PhotoQt.org}>
\item Development page: <\URL{https://gitlab.com/luspi/photoqt/issues}>
\end{itemize}

\section{See also}
For more information about the features of the application, visit the
online manual:\\
<\URL{https://photoqt.org/man/}>

\section{Copyright}
\noindent
\Prog{PhotoQt} - Copyright \copyright\ 2011-2018 \textbf{Lukas Spies}\\
This manual page was written and translated to Spanish by Miguel Molina
<\Email{mmolina.unphysics@gmail.com}>. Permission is granted to copy,
distribute and/or modify this document under the terms of the GNU General
Public License, Version 2 any later version published by the Free Software
Foundation.

\LatexManEnd

\end{document}
