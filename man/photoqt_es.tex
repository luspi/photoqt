% Info: http://www.informatik-vollmer.de/software/latex2man-html.php
\documentclass[10pt]{article}
% Paquetes
\usepackage[latin1]{inputenc}
\usepackage{ae}
\usepackage{aecompl}
\usepackage[spanish]{babel}
\usepackage[pdftex,bookmarksopen,bookmarksnumbered]{hyperref}
\usepackage[nofancy]{latex2man}


\begin{document}

% Fecha y Version para LaTeX:
% $ date '+%Y-%m-%d'
\setDate{2018-06-20}
\setVersion{1.7.1}

\begin{Name}{1}{photoqt}{Lukas Spies}{\pkgname-\pkgversion}{photoqt -- Un simple y atractivo Visor de Im�genes}
\Prog{PhotoQt} - Un simple y atractivo Visor de Im�genes
\end{Name}

\section{Sinopsis}
\noindent
\Prog{photoqt} \oArg{opciones} \oArg{nombre de archivo}

\section{Descripci�n}
\noindent
\Prog{PhotoQt} es un visor de im�genes r�pido y atractivo desarrollado
para diferentes plataformas. La aplicaci�n usa distintos motores de
visualizaci�n, por lo que soporta m�s de 80 tipos de archivos de imagen.
Adem�s, permite un manejo y gesti�n de im�genes de forma r�pida e
intuitiva, logrando interactuar con el usuario a trav�s de accesos
r�pidos totalmente personalizables.\MANbr
Vistas en miniatura, presentaciones en diapositivas, informaci�n de la
imagen y otras funciones hacen que \Prog{PhotoQt} sea una poderosa
elecci�n como el programa predeterminado para visualizaci�n de im�genes
de su sistema.

\section{Opciones}
Opciones gen�ricas:
\begin{Description}
\item[\Opt{-h, --help}] Mostrar esta ayuda.
\item[\Opt{-v, --version}] Mostrar versi�n.
\end{Description}
Interacci�n con PhotoQt:
\begin{Description}
\item[\Opt{-o, --open}] Hacer que PhotoQt solicite un nuevo archivo.
\item[\Opt{-s, --show}] Muestra PhotoQt (no hace nada si est� en ejecuci�n).
\item[\Opt{--hide}] Ocultar PhotoQt en la bandeja de entrada (no hace
nada si est� oculto).
\item[\Opt{-t, --toggle}]  Alternar PhotoQt: ocultar PhotoQt si est�
visible, mostrar si est� oculto.
\item[\Opt{--thumbs, --no-thumbs}] Activar/Desactivar miniaturas.
\end{Description}
Opciones solamente al inicio:
\begin{Description}
\item[\Opt{--start-in-tray}] Iniciar PhotoQt oculto en la bandeja de entrada.
\item[\Opt{--standalone}] Crea sesi�n PhotoQt independiente, m�ltiples
instancias pero sin interacci�n remota.
\end{Description}
Opciones Generales:
\begin{Description}
\item[\Opt{--debug, --no-debug}] Activar/Desactivar mensajes de depuraci�n.
\item[\OptoArg{--export}{nombre de archivo}] Exportar configuraci�n hacia
un archivo.
\item[\OptoArg{--import}{nombre de archivo}] Importar configuraci�n desde
un archivo.
\end{Description}
Argumentos:
\begin{Description}
\item[\oArg{nombre de archivo}] Abrir archivo con PhotoQt.
\end{Description}

\section{Archivos}
\noindent
La aplicaci�n crea y usa los siguientes directorios de configuraci�n:
\begin{description}
\item[\$HOME/.config/PhotoQt] Ajustes personales en plataformas Linux.
Use la interfaz GUI en 'Ajustes' para modificarlas.
\item[\%USERPROFILE\%\Bs AppData\Bs Local\Bs PhotoQt] Ajustes personales en
plataformas Windows. Use la interfaz GUI en 'Ajustes' para modificarlas.
\end{description}

\section{Autor}
\noindent
Lukas Spies, desarrollador, <\Email{Lukas@PhotoQt.org}>.\\
P�gina web del proyecto: <\URL{http://photoqt.org/}>.

\section{Bugs}
\noindent
Puede enviar reporte de errores por medio de:
\begin{itemize}[*]
\item Email: <\Email{Lukas@PhotoQt.org}>
\item P�gina de desarrollo: <\URL{https://gitlab.com/luspi/photoqt/issues}>
\end{itemize}

\section{Ver tambi�n}
Para m�s informaci�n acerca de las caracter�sticas de la aplicaci�n,
visite el manual en l�nea:\\
<\URL{https://photoqt.org/man/}>

\section{Copyright}
\noindent
\Prog{PhotoQt} - Copyright \copyright\ 2011-2018 \textbf{Lukas Spies}\\
Esta p�gina fue escrita y traducida al espa�ol por Miguel Molina
<\Email{mmolina.unphysics@gmail.com}>. Se permite copiar, distribuir y/o
modificar este documento bajo los t�rminos de GNU Licencia P�blica
General Versi�n 2 o cualquier versi�n posterior publicada por la Free
Fundaci�n de software.

\LatexManEnd

\end{document}
